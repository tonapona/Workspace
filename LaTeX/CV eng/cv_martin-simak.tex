%============================================================================%
%
%	DOCUMENT DEFINITION
%
%============================================================================%

\documentclass[10pt]{article}	


%----------------------------------------------------------------------------------------
%	ENCODING
%----------------------------------------------------------------------------------------

% we use utf8 since we want to build from any machine
\usepackage[utf8]{inputenc}
\usepackage[T1]{fontenc}
\usepackage[USenglish]{isodate}
\usepackage{fancyhdr}
\usepackage[numbers]{natbib}

%----------------------------------------------------------------------------------------
%	LOGIC
%----------------------------------------------------------------------------------------

% provides \isempty test
\usepackage{xstring, xifthen}
\usepackage{enumitem}
\usepackage[british]{babel}
\usepackage{pdfpages}
\usepackage{changepage}
%----------------------------------------------------------------------------------------
%	FONT BASICS
%----------------------------------------------------------------------------------------

% some tex-live fonts - choose your own

%\usepackage[defaultsans]{droidsans}
%\usepackage[default]{comfortaa}
%\usepackage{cmbright}
\usepackage[default]{raleway}
%\usepackage{fetamont}
%\usepackage[default]{gillius}
%\usepackage[light,math]{iwona}
%\usepackage[thin]{roboto} 

% set font default
\renewcommand*\familydefault{\sfdefault} 	
\usepackage[T1]{fontenc}

% more font size definitions
\usepackage{moresize}

%----------------------------------------------------------------------------------------
%	FONT AWESOME ICONS
%---------------------------------------------------------------------------------------- 

% include the fontawesome icon set
\usepackage{fontawesome}

% use to vertically center content
% credits to: http://tex.stackexchange.com/questions/7219/how-to-vertically-center-two-images-next-to-each-other
\newcommand{\vcenteredinclude}[1]{\begingroup
\setbox0=\hbox{\includegraphics{#1}}%
\parbox{\wd0}{\box0}\endgroup}
\newcommand{\tab}[1]{\hspace{.2\textwidth}\rlap{#1}}
% use to vertically center content
% credits to: http://tex.stackexchange.com/questions/7219/how-to-vertically-center-two-images-next-to-each-other
\newcommand*{\vcenteredhbox}[1]{\begingroup
\setbox0=\hbox{#1}\parbox{\wd0}{\box0}\endgroup}

% icon shortcut
\newcommand{\icon}[3] { 							
	\makebox(#2, #2){\textcolor{maincol}{\csname fa#1\endcsname}}
}	


% icon with text shortcut
\newcommand{\icontext}[4]{ 						
	\vcenteredhbox{\icon{#1}{#2}{#3}}  \hspace{2pt}  \parbox{0.9\mpwidth}{\textcolor{#4}{#3}}
}

% icon with website url
\newcommand{\iconhref}[5]{ 						
    \vcenteredhbox{\icon{#1}{#2}{#5}}  \hspace{2pt} \href{#4}{\textcolor{#5}{#3}}
}

% icon with email link
\newcommand{\iconemail}[5]{ 						
    \vcenteredhbox{\icon{#1}{#2}{#5}}  \hspace{2pt} \href{mailto:#4}{\textcolor{#5}{#3}}
}

%----------------------------------------------------------------------------------------
%	PAGE LAYOUT  DEFINITIONS
%----------------------------------------------------------------------------------------

% page outer frames (debug-only)
% \usepackage{showframe}		

% we use paracol to display breakable two columns
\usepackage{paracol}
\usepackage{tikzpagenodes}
\usetikzlibrary{calc}
\usepackage{lmodern}
\usepackage{multicol}
\usepackage{lipsum}
\usepackage{atbegshi}
% define page styles using geometry
\usepackage[a4paper]{geometry}

% remove all possible margins
\geometry{top=1cm, bottom=1cm, left=1cm, right=1cm}

\usepackage{fancyhdr}
\pagestyle{empty}

% space between header and content
% \setlength{\headheight}{0pt}

% indentation is zero
\setlength{\parindent}{0mm}

%----------------------------------------------------------------------------------------
%	TABLE /ARRAY DEFINITIONS
%---------------------------------------------------------------------------------------- 

% extended aligning of tabular cells
\usepackage{array}

% custom column right-align with fixed width
% use like p{size} but via x{size}
\newcolumntype{x}[1]{%
>{\raggedleft\hspace{0pt}}p{#1}}%


%----------------------------------------------------------------------------------------
%	GRAPHICS DEFINITIONS
%---------------------------------------------------------------------------------------- 

%for header image
\usepackage{graphicx}

% use this for floating figures
% \usepackage{wrapfig}
% \usepackage{float}
% \floatstyle{boxed} 
% \restylefloat{figure}

%for drawing graphics		
\usepackage{tikz}			
\usepackage{ragged2e}	
\usetikzlibrary{shapes, backgrounds,mindmap, trees}

%----------------------------------------------------------------------------------------
%	Color DEFINITIONS
%---------------------------------------------------------------------------------------- 
\usepackage{transparent}
\usepackage{color}

% primary color
\definecolor{maincol}{RGB}{ 64,64,64}

% accent color, secondary
% \definecolor{accentcol}{RGB}{ 250, 150, 10 }

% dark color
\definecolor{darkcol}{RGB}{ 70, 70, 70 }

% light color
\definecolor{lightcol}{RGB}{245,245,245}

\definecolor{accentcol}{RGB}{59,77,97}



% Package for links, must be the last package used
\usepackage[hidelinks]{hyperref}

% returns minipage width minus two times \fboxsep
% to keep padding included in width calculations
% can also be used for other boxes / environments
\newcommand{\mpwidth}{\linewidth-\fboxsep-\fboxsep}
	


%============================================================================%
%
%	CV COMMANDS
%
%============================================================================%

%----------------------------------------------------------------------------------------
%	 CV LIST
%----------------------------------------------------------------------------------------

% renders a standard latex list but abstracts away the environment definition (begin/end)
\newcommand{\cvlist}[1] {
	\begin{itemize}{#1}\end{itemize}
}

%----------------------------------------------------------------------------------------
%	 CV TEXT
%----------------------------------------------------------------------------------------

% base class to wrap any text based stuff here. Renders like a paragraph.
% Allows complex commands to be passed, too.
% param 1: *any
\newcommand{\cvtext}[1] {
	\begin{tabular*}{1\mpwidth}{p{0.98\mpwidth}}
		\parbox{1\mpwidth}{#1}
	\end{tabular*}
}
\newcommand{\cvtextsmall}[1] {
	\begin{tabular*}{0.8\mpwidth}{p{0.8\mpwidth}}
		\parbox{0.8\mpwidth}{#1}
	\end{tabular*}
}
%----------------------------------------------------------------------------------------
%	CV SECTION
%----------------------------------------------------------------------------------------

% Renders a a CV section headline with a nice underline in main color.
% param 1: section title
\newcommand{\cvsection}[1] {
	\vspace{14pt}
	\cvtext{
		\textbf{\LARGE{\textcolor{darkcol}{#1}}}\\[-4pt]
		\textcolor{accentcol}{ \rule{0.2\textwidth}{1.5pt} } \\
	}
}

\newcommand{\cvsectionsmall}[1] {
	\vspace{14pt}
	\cvtext{
		\textbf{\Large{\textcolor{darkcol}{#1}}}\\[-4pt]
		\textcolor{accentcol}{ \rule{0.2\textwidth}{1.5pt} } \\
	}
}

\newcommand{\cvheadline}[1] {
	\vspace{16pt}
	\cvtext{
		\textbf{\Huge{\textcolor{accentcol}{#1}}}\\[-4pt]
		 
	}
}

\newcommand{\cvsubheadline}[1] {
	\vspace{16pt}
	\cvtext{
		\textbf{\huge{\textcolor{darkcol}{#1}}}\\[-4pt]
		 
	}
}
%----------------------------------------------------------------------------------------
%	META SKILL
%----------------------------------------------------------------------------------------

% Renders a progress-bar to indicate a certain skill in percent.
% param 1: name of the skill / tech / etc.
% param 2: level (for example in years)
% param 3: percent, values range from 0 to 1
\newcommand{\cvskill}[3] {
	\begin{tabular*}{1\mpwidth}{l @{\extracolsep{\fill}} r}
 		\textcolor{black}{\textbf{#1}} & \textcolor{maincol}{#2}\\
	\end{tabular*}%
	
	\hspace{4pt}
	\begin{tikzpicture}[scale=1,rounded corners=2pt,very thin]
		\fill [lightcol] (0,0) rectangle (1\mpwidth, 0.15);
		\fill [accentcol] (0,0) rectangle (#3\mpwidth, 0.15);
  	\end{tikzpicture}%
}


%----------------------------------------------------------------------------------------
%	 CV EVENT
%----------------------------------------------------------------------------------------

% Renders a table and a paragraph (cvtext) wrapped in a parbox (to ensure minimum content
% is glued together when a pagebreak appears).
% Additional Information can be passed in text or list form (or other environments).
% the work you did
% param 1: time-frame i.e. Sep 14 - Jan 15 etc.
% param 2:	 event name (job position etc.)
% param 3: Customer, Employer, Industry
% param 4: Short description
% param 5: work done (optional)
% param 6: technologies include (optional)
% param 7: achievements (optional)
\newcommand{\cvevent}[7] {
	
	% we wrap this part in a parbox, so title and description are not separated on a pagebreak
	% if you need more control on page breaks, remove the parbox
	\parbox{\mpwidth}{
		\begin{tabular*}{1\mpwidth}{p{0.66\mpwidth}  r}
	 		\textcolor{black}{\textbf{#2}} & \colorbox{accentcol}{\makebox[0.32\mpwidth]{\textcolor{white}{\textbf{#1}}}} \\
			\textcolor{maincol}{#3} & \\
		\end{tabular*}\\[8pt]
	
		\ifthenelse{\isempty{#4}}{}{
			\cvtext{#4}\\
		}
	}
	\vspace{14pt}
}


%----------------------------------------------------------------------------------------
%	 CV META EVENT
%----------------------------------------------------------------------------------------

% Renders a CV event on the sidebar
% param 1: title
% param 2: subtitle (optional)
% param 3: customer, employer, etc,. (optional)
% param 4: info text (optional)
\newcommand{\cvmetaevent}[4] {
	\textcolor{maincol} { \cvtext{\textbf{\begin{flushleft}#1\end{flushleft}}}}

	\ifthenelse{\isempty{#2}}{}{
	\textcolor{black} {\cvtext{\textbf{#2}} }
	}

	\ifthenelse{\isempty{#3}}{}{
		\cvtext{{ \textcolor{maincol} {#3} }}\\
	}

	\cvtext{#4}\\[14pt]
}

%---------------------------------------------------------------------------------------
%	QR CODE
%----------------------------------------------------------------------------------------

% Renders a qrcode image (centered, relative to the parentwidth)
% param 1: percent width, from 0 to 1
\newcommand{\cvqrcode}[1] {
	\begin{center}
		\includegraphics[width={#1}\mpwidth]{qrcode}
	\end{center}
}


% HEADER AND FOOOTER 
%====================================
\newcommand\Header[1]{%
\begin{tikzpicture}[remember picture,overlay]
\fill[accentcol]
  (current page.north west) -- (current page.north east) --
  ([yshift=50pt]current page.north east|-current page text area.north east) --
  ([yshift=50pt,xshift=-3cm]current page.north|-current page text area.north) --
  ([yshift=10pt,xshift=-5cm]current page.north|-current page text area.north) --
  ([yshift=10pt]current page.north west|-current page text area.north west) -- cycle;
\node[font=\sffamily\bfseries\color{white},anchor=west,
  xshift=0.7cm,yshift=-0.32cm] at (current page.north west)
  {\fontsize{12}{12}\selectfont {#1}};
\end{tikzpicture}%
}

\newcommand\Footer[1]{%
\begin{tikzpicture}[remember picture,overlay]
\fill[lightcol]
  (current page.south east) -- (current page.south west) --
  ([yshift=-80pt]current page.south east|-current page text area.south east) --
  ([yshift=-80pt,xshift=-6cm]current page.south|-current page text area.south) --
  ([xshift=-2.5cm,yshift=-10pt]current page.south|-current page text area.south) --	
  ([yshift=-10pt]current page.south east|-current page text area.south east) -- cycle;
\node[yshift=0.32cm,xshift=9cm] at (current page.south) {\fontsize{10}{10}\selectfont \textbf{\thepage}};
\end{tikzpicture}%
}


%=====================================
%============================================================================%
%
%
%
%	DOCUMENT CONTENT
%
%
%
%============================================================================%
\begin{document}

\columnratio{0.31}
\setlength{\columnsep}{2.2em}
\setlength{\columnseprule}{4pt}
\colseprulecolor{white}

\pagenumbering{gobble}


% LEBENSLAUF HIERE
\AtBeginShipoutFirst{\Header{CV}\Footer{1}}
\AtBeginShipout{\AtBeginShipoutAddToBox{\Header{CV}\Footer{2}}}

\newpage

\colseprulecolor{lightcol}
\columnratio{0.31}
\setlength{\columnsep}{2.2em}
\setlength{\columnseprule}{4pt}
\begin{paracol}{2}


\begin{leftcolumn}
%---------------------------------------------------------------------------------------
%	META IMAGE
%----------------------------------------------------------------------------------------
\includegraphics[width=0.85\linewidth]{resources/portrait.jpg}
        
%---------------------------------------------------------------------------------------
%	META SKILLS
%---------------------------------------------------------------------------------------
\fcolorbox{white}{white}{\begin{minipage}[c][1.5cm][c]{1\mpwidth}
\LARGE{\textbf{\textcolor{maincol}{Martin Šimák}}} \\[2pt]
\normalsize{ \textcolor{maincol} {Electrical engineering student} }
\end{minipage}} \\
\icontext{CaretRight}{12}{10 February 1999, Czechia}{black}\\[6pt]
\icontext{CaretRight}{12}{ISTQB Certified Tester}{black}\\[6pt]
\icontext{CaretRight}{12}{European driving license class B}{black}

\cvsection{Skills}

\cvskill{TeX} {advanced} {0.8} \\[-2pt]

\cvskill{Linux (Arch/Debian)} {advanced} {0.7} \\[-2pt]

\cvskill{AWR MO} {advanced} {0.6} \\[-2pt]

\cvskill{Matlab} {advanced} {0.6} \\[-2pt]

\cvskill{CST Studio Suite} {intermediate} {0.5} \\[-2pt]

\cvskill{Python} {intermediate} {0.5} \\[-2pt]

\cvskill{C} {intermediate} {0.4} \\[-2pt]

\cvskill{C++} {beginner} {0.3} \\[10pt]

\cvskill{Czech} {native} {1} \\[-2pt]

\cvskill{English} {C1/C2} {0.9} \\[-2pt]

\cvskill{German} {A1} {0.25} \\[-2pt]% \\

%---------------------------------------------------------------------------------------
%	CONTACT
%---------------------------------------------------------------------------------------
\newpage\cvsection{Contact}

\icontext{MobilePhone}{16}{+420 602 574 563}{black}\\[6pt]
\iconemail{Envelope}{16}{martin.simaak@gmail.com}{martin.simaak@gmail.com}{black}\\[6pt]
\iconhref{Github}{16}{github.com/simakma5}{https://github.com/simakma5/Workspace}{black}\\[6pt]

	
%\cvqrcode{0.3}

\end{leftcolumn}
\begin{rightcolumn}
%---------------------------------------------------------------------------------------
%	TITLE  HEADER
%----------------------------------------------------------------------------------------


%---------------------------------------------------------------------------------------
%	PROFILE
%----------------------------------------------------------------------------------------
\cvsection{Profile}

\cvtext{I am an electrical engineering student with a major focus on radio-frequency engineering. Currently, I also work on a microwave research project for the faculty and as a test engineer for a prominent telecommunication company. Apart from my career interests, I have played the piano for 12 years, hence I have a strong passion for music, and I enjoy most kinds of sports.}


%---------------------------------------------------------------------------------------
%	EDUCATION
%----------------------------------------------------------------------------------------
\vspace{20pt}
\cvsection{Education}

\cvevent
{2022 - present}
{Electronics and Communications}
{FEE CTU in Prague}
{I have chosen this master's degree programme with a specialization in Radio Communications and Systems where I further deepen my knowledge in radio-frequency engineering, such as radar systems and wireless communications.}
\vfill\null

\cvevent
{2018 - 2022}
{Open Electronic Systems}
{FEE CTU in Prague}
{For my bachelor's degree studies, I had chosen this programme focused on the fundamentals of electrical engineering from the theoretical viewpoint which offered me a robust understanding of the principles required for further research.}
\vfill\null

\cvevent
{2014 - 2018}
{Maturity exam}
{Pierre de Coubertin Grammar School in Tábor}
{}
\vfill\null

\vspace{-20pt}


%---------------------------------------------------------------------------------------
%	WORK EXPERIENCE
%----------------------------------------------------------------------------------------

\cvsection{Employment history}

\cvevent
    {2023-present}
	{Research assistant}
	{FEE CTU in Prague}
	{Thanks to my good results during the course of microwave circuit design, I have been offered participation in a microwave research project for a Czech company Argotech. This involves further improving my understanding of microwave circuits, performing 3D EM analyses, calibrating measurement devices etc.}
	\vfill\null

\cvevent
    {2022-present}
	{Junior Test Engineer}
	{2N TELEKOMUNIKACE a.s.}
	{Working in a modern, international telecommunication company helped me get a better understanding of software development and engineering practices. Nowadays, I specialize in Python scripting for automated software testing.}
	\vfill\null

\cvevent
    {2021}
	{Mathematics and physics lecturer}
	{FEE CTU in Prague}
	{During the summer semester of the academic year 2020/2021, I co-lead the courses in high-school mathematics and physics for students of the faculty.}
	\vfill\null

\cvevent
{2021, 2020}
	{Junior Helpdesk Operator}
	{S\&T CZ s.r.o.}
	{This summer job involved working for an extern company delivering technical support for the Ministry of Foreign Affairs of the Czech Republic.}
	\vfill\null

\cvevent
{2018, 2019}
	{Tour guide}
	{Castle Orlík, Castle Konopiště}
	{During this summer job, I have conducted large groups of people in both Czech and English, which helped me improve my public presentation in both languages.}
	\vfill\null

\cvevent
    {2015, 2016, 2017}
	{Fitter}
	{EFAFLEX-CZ s.r.o.}
	{From my first regular summer job, I gained the practical experience of a manual labourer in a mechanical company.}
	\vfill\null

\end{rightcolumn}
\end{paracol}


\end{document}